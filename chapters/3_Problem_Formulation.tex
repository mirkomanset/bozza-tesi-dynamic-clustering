\chapter{Problem formulation}\label{ch:problem_formulation}

This chapter establishes the theoretical foundation of the thesis by formally
defining the problem space. It focuses on the formulation of clustering in the
context of evolving data streams, where data distributions are non-stationary
and subject to various forms of drift over time. Traditional clustering
methods, which assume access to a fixed dataset and a static distribution,
prove inadequate in such scenarios.

To address these challenges, the concept of \emph{streaming clustering} is
introduced, where data points are processed incrementally. However, streaming
alone is insufficient when the structure of the data itself evolves. Therefore,
attention is extended to \emph{dynamic clustering}, a framework that performs
clustering in real time while also tracking and interpreting the
transformations of clusters over time.

\section{Streaming Clustering}\label{sec:prob_streaming_clustering}

A \emph{data stream} is a potentially infinite sequence of multi-dimensional
data points:
\begin{equation}
    \mathcal{X} = \bigl\{\mathbf{x}^t\bigr\}_{t=1}^{\infty},
\end{equation}
where each $\mathbf{x}^t \in \mathbb{R}^d$ represents the $d$-dimensional observation received at timestamp $t$.

The goal of streaming clustering is to incrementally partition the incoming
data into coherent groups, while meeting strict memory and computational
constraints. At any time step $t$, the algorithm produces a clustering result:
\begin{equation}
    C^t = \{C_1^t, C_2^t, \dots, C_{k^t}^t\},
\end{equation}
where $C^t$ is the set of clusters at time $t$, $k^t$ is the number of clusters, and each $C_i^t$ contains a subset of the observed data up to time $t$.

Streaming clustering is considered in the context of non-stationary
environments, where the data-generating distribution evolves over time. To make
this setting both tractable and realistic, the following assumptions are
adopted regarding the nature of distributional change:

\begin{itemize}
    \item The evolution of the data is assumed to be \emph{gradual and smooth}, rather
          than abrupt.
    \item The data stream exhibits \emph{recurrence}, meaning that patterns observed in
          the past may reappear in the future.
\end{itemize}

To formalize this setting, the notion of a \textbf{regular drift sequence} is
employed. Let $ S = \{P_1, \dots, P_n\} $ denote a sequence of probability
distributions generating the data stream. Given a distance metric $ d(\cdot,
\cdot) $ and a vector $ \boldsymbol{\Delta}_n = (\Delta_1, \dots, \Delta_n)
\in \mathbb{R}_{\ge 0}^n $, the sequence $ S $ is said to be a regular drift
sequence if the following conditions hold:

\begin{itemize}
    \item The drift between each $ P_i $ and the current distribution $ P_n $ is
          bounded above by $ \Delta_i $, i.e., $ d(P_i, P_n) \le \Delta_i $.
    \item The sequence $ \Delta_1, \dots, \Delta_n $ is non-increasing.
    \item Drift evolves smoothly, i.e., there exists a constant $ c $ such that $
          |\Delta_{i-1} - \Delta_i| \le c $ for all $ i \in \{2, \dots, n\} $.
\end{itemize}

A more general and realistic scenario arises when the full distributional
evolution is not globally regular but consists of segments that each satisfy
the regular drift conditions. In this case, the sequence $ S = \{P_a, \dots,
P_b\} $ is defined as a \textbf{piecewise regular drift sequence} if it can be
partitioned into a series of consecutive, non-overlapping regular drift
segments $ \{S_j\}_{j \in \mathbb{N}} $. Additionally, it is required that
for any pair of consecutive distributions $ P_k \in S_j $ and $ P_{k+1} \in
S_{j+1} $ (from adjacent segments), the drift is bounded:

\[
    d(P_k, P_{k+1}) \le \Delta,
\]

for some constant $ \Delta $. This condition constrains the severity of
changes between successive phases of the stream and ensures that the clustering
algorithm remains stable and responsive to evolving structure.

Under this formulation, streaming clustering algorithms are required to process
incoming data points in real time and update the clustering solution
accordingly. The resulting clustering $ C^t $ may evolve significantly in
response to concept drift, noise, or the emergence of new structures.

\section{Tracking Clusters}\label{sec:prob_tracking_clusters}
In non-stationary data streams, clustering structures evolve over time due to
changes in the underlying data distribution. Consequently, the clustering model
$C^t$ at time $t$ may differ substantially from $C^\tau$ at a later time $\tau
    > t$. To analyze such dynamics, it is essential to model and track how
individual clusters change across time.

Given a sequence of clustering models $\{C^0, C^1, \dots, C^T\}$ over time,
where each $C^t = \{C_1^t, \dots, C_{k_t}^t\}$ represents a set of clusters at
time $t$, the goal is to identify and categorize transitions between clusters
over time.

As discussed by Oliveira et Gama~\cite{mec}, clusters may undergo several types
of transitions:

\begin{itemize}
    \item \textbf{Appearance}: A new cluster is formed that did not exist previously.
    \item \textbf{Disappearance}: A previously existing cluster ceases to exist.
    \item \textbf{Split}: A single cluster divides into multiple clusters.
    \item \textbf{Merge}: Multiple clusters combine into a single cluster.
    \item \textbf{Survival}: A cluster remains relatively unchanged over time.
\end{itemize}

To formally define cluster transitions, it is necessary to introduce the notion
of \textbf{overlap} between clusters at different timestamps. Given a cluster $
    C_i^t \in C^t $ and a cluster $ C_j^\tau \in C^\tau $, with $ t < \tau $, these
clusters are said to \emph{overlap}, denoted as

\begin{equation}
    C_i^t \circledast C_j^\tau,
\end{equation}

if they exhibit a non-negligible intersection.

Formally, let $\mathcal{S}(C_i^t, C_j^{\tau})$ be a generic \emph{overlap score
    function} that quantifies the degree of similarity or intersection between the
two clusters. Then,
\begin{equation}
    C_i^t \circledast C_j^\tau \iff \mathcal{S}(C_i^t, C_j^{\tau}) \ge \varepsilon,
\end{equation}
where $\varepsilon \in \mathbb{R}$ is a predefined threshold that determines the
minimum level of overlap required to consider the clusters as related.

The specific form of $\mathcal{S}$ depends on the cluster representation being
employed, common examples include:

\begin{itemize}
    \item \textbf{Enumeration-based Representation:} clusters are explicitly
          defined as the sets of data points they contain. Overlap between two
          clusters is determined by the presence of at least one shared instance:
          \begin{equation}
              C_i^t \circledast C_j^\tau \iff \exists \mathbf{x} \in \mathbb{R}^n \text{ such that } \mathbf{x} \in C_i^t \land \mathbf{x} \in C_j^\tau,
          \end{equation}

    \item \textbf{Spherical Model Representation:} clusters are approximated as hyperspheres characterized by their centroid $\boldsymbol{\mu}$ and radius $r$. Overlap is defined as follows:
          \begin{equation}
              C_i^t \circledast C_j^\tau \iff ||\boldsymbol{\mu}_i^t - \boldsymbol{\mu}_j^\tau||_2 \leq r_i^t + r_j^\tau,
          \end{equation}
          where $\boldsymbol{\mu}_i^t$ and $\boldsymbol{\mu}_j^\tau$ denote the centroids of clusters $C_i^t$ and $C_j^\tau$, respectively, $r_i^t$ and $r_j^\tau$ represent their radii.
\end{itemize}

Table~\ref{table:cluster_transitions} summarizes the formal definitions of typical cluster
transitions based on the concept of overlap:

\begin{table}[H]
    \centering
    {\footnotesize
        \begin{tabular}{|l|c|p{9cm}|}
            \hline
            \textbf{Transition} & \textbf{Notation}                                 & \textbf{Formal Definition}                                       \\
            \hline
            Appearance          & $\emptyset^t \rightarrow C_j^\tau$                  &
            $\nexists C_i^t \in C^t: C_i^t \circledast C_j^{\tau}$                                                                                     \\
            \hline
            Disappearance       & $C_i^t \rightarrow \emptyset^t$                     & $\nexists C_j^{\tau} \in C^{\tau}: C_i^t \circledast C_j^{\tau}$ \\
            \hline
            Split               & $C_i^t \rightarrow \{C_1^\tau, \dots, C_v^\tau\}$ & $\forall j \in \{1, \dots, v\}: C_i^t  \circledast C_j^{\tau}$   \\
            \hline
            Merge               & $\{C_1^t, \dots, C_u^t\} \rightarrow C_j^\tau$    & $\forall i \in \{1, \dots, u\}: C_i^t  \circledast C_j^{\tau}$   \\
            \hline
            Survival            & $C_i^t \rightarrow C_j^\tau$                      & $C_i^t \circledast C_j^\tau \land
                \nexists\, C_u^t \in C^t \setminus \{C_i^t\}: C_u^t \circledast C_j^\tau \land
            \nexists\, C_v^\tau \in C^\tau \setminus \{C_j^\tau\}: C_i^t \circledast C_v^\tau$                                                         \\
            \hline
        \end{tabular}
    }
    \caption{Formal definitions of cluster transitions.}\label{table:cluster_transitions}
\end{table}

In contrast to the external transitions described above, which involve
structural changes between distinct clusters over time, \textbf{internal
    transitions} refer to the transformations that occur \emph{within} a single
surviving cluster. These transitions are analyzed only when a cluster persists
across time (i.e., it undergoes a \emph{survival} transition), and they
characterize the evolution of the cluster's internal properties.

Internal transitions typically involve comparisons of specific characteristics
of a cluster between time steps, such as:

\begin{itemize}
    \item \textbf{Expansion or Shrinkage}: Changes in the cardinality (number of
          elements) or spatial extent of a cluster, indicating that the cluster has
          grown or contracted.
    \item \textbf{Density Variation}: Alterations in the density of a cluster,
          often defined in terms of the number of points per unit volume, which may
          suggest dispersion or compaction of the cluster.
    \item \textbf{Centroid Shift}: Movement of the cluster centroid over time,
          reflecting a displacement in the central location of the cluster within the
          feature space.
\end{itemize}

Although internal transitions provide additional insight into the temporal
dynamics of clusters, they are not explicitly addressed in this thesis. The
primary focus is on external transitions, which are more directly related to
the structural evolution of the clustering model. Nevertheless, internal
transitions can be directly inferred by examining the state of a surviving
cluster before and after a time step, allowing for indirect analysis when
needed.

Tracking clusters in dynamic streaming environments presents several key
challenges. One of the foremost difficulties is dealing with data drift and
evolution, where the underlying data distribution may shift gradually or
abruptly, causing clusters to move, emerge, disappear, merge, or split in
complex and unpredictable ways. The presence of noise and outliers further
complicates the problem by potentially masking genuine cluster transitions or
generating false alarms. Ambiguities often arise when clusters overlap or lie
in close proximity, making it challenging to precisely define cluster
correspondences and detect transitions accurately. Finally, the absence of
ground truth labels in streaming contexts makes it difficult to evaluate the
correctness of detected transitions and adapt algorithms to evolving cluster
structures.

\section{Dynamic Clustering}\label{sec:dynamic_clustering}

Dynamic clustering aims to provide a holistic framework for understanding and
responding to the evolution of data distributions in non-stationary
environments. By combining streaming clustering with temporal cluster tracking,
it enables both real-time analysis and retrospective insight into how clusters
emerge, change, and disappear over time.

The objective is not only to cluster incoming data efficiently, but also to
interpret and explain the structural changes in the data. This is particularly
valuable in applications where understanding data dynamics is as important as
making accurate predictions or decisions.

Dynamic clustering supports a range of analytical goals, including:

\begin{itemize}
    \item \textbf{Drift Explainability:} By analyzing the history and transformation of
          clusters, one can identify when and how data drift occurs, and potentially
          relate it to changes in user behavior, system state, or external conditions.

    \item \textbf{Concept Monitoring and Lifecycle Analysis:} Tracking clusters over time
          provides a basis for monitoring the emergence, evolution, and retirement of
          underlying concepts or patterns in the data.

    \item \textbf{Anomaly and Event Detection:} Sudden transitions, such as abrupt merges,
          splits, or disappearances, can signal significant events or anomalies in the data stream,
          such as system faults or behavioral shifts.

    \item \textbf{Interpretability and Visualization:} Visualizing cluster evolution as
          trajectories, trees, or timelines helps domain experts make sense of complex temporal
          dynamics in an intuitive manner.
\end{itemize}

By treating clustering as a dynamic process rather than a static task, this
framework enables richer, more actionable insights in data-driven systems that
operate under change.