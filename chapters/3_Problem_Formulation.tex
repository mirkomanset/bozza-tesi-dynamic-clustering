\chapter{Problem formulation}\label{ch:problem_formulation}

\section{Streaming Clustering}\label{sec:prob_streaming_clustering}
A \textit{data stream} is a potentially infinite sequence of multi-dimensional
data points:
\begin{equation}
    \chi = \{x^t\}_{t=1}^{\infty}, \quad x^t \in \mathbb{R}^n
\end{equation}
where $x^t$ denotes the $n$-dimensional observation received at timestamp $t$.
Due to the unbounded nature of data streams and limited memory or computation
constraints, traditional batch clustering methods are not applicable in this
setting.

The goal of streaming clustering is to partition the incoming data points into
groups in a timely and resource-efficient manner. At any given time $t$, a
clustering algorithm processes the stream incrementally and produces a
clustering result:
\begin{equation}
    C^t = \{c_1^t, c_2^t, \dots, c_{k^t}^t\}
\end{equation}
where:
\begin{itemize}
    \item $C^t$ denotes the clustering obtained at time $t$,
    \item $k^t$ is the number of clusters at time $t$,
    \item $c_i^t$ is the $i$-th cluster at time $t$, defined as a subset of data
          points observed up to time $t$.
\end{itemize}

Streaming clustering methods must satisfy the following constraints:

\begin{enumerate}
    \item \textbf{Incrementality:} The algorithm must be able to update the clustering
          with each new incoming data point or small batch, without access to the full
          historical data.

    \item \textbf{Bounded Memory:} The algorithm must operate within limited memory,
          often without storing all previous data points.

    \item \textbf{Computational Efficiency:} The algorithm must produce clustering
          updates within tight time constraints to maintain real-time responsiveness.

    \item \textbf{Adaptability:} The algorithm should be capable of handling non-stationary
          data distributions, as the characteristics of the stream may change over time.
\end{enumerate}

At each time step, the algorithm processes new data and outputs an updated
clustering $C^t$. The clustering may change significantly over time due to
drift, noise, or the arrival of new patterns in the stream.

\section{Tracking Clusters}\label{sec:prob_tracking_clusters}
In non-stationary data streams, clustering structures evolve over time due to
changes in the underlying data distribution. Consequently, the clustering model
$C^t$ at time $t$ may differ substantially from $C^\tau$ at a later time $\tau
    > t$. To analyze such dynamics, it is essential to model and track how
individual clusters change across time.

Given a sequence of clustering models $\{C^0, C^1, \dots, C^T\}$ over time,
where each $C^t = \{c_1^t, \dots, c_{k_t}^t\}$ represents a set of clusters at
time $t$, the goal is to identify and categorize transitions between clusters
over time.

Clusters may undergo several types of transitions:

\begin{itemize}
    \item \textbf{Appearance}: A new cluster is formed that did not exist previously.
    \item \textbf{Disappearance}: A previously existing cluster ceases to exist.
    \item \textbf{Split}: A single cluster divides into multiple clusters.
    \item \textbf{Merge}: Multiple clusters combine into a single cluster.
    \item \textbf{Survival}: A cluster remains relatively unchanged over time.
\end{itemize}

To formally define cluster transitions, it is necessary to introduce the notion
of \textbf{overlap} between clusters at different timestamps. Given a cluster $
    c_i^t \in C^t $ and a cluster $ c_j^\tau \in C^\tau $, with $ t < \tau $, these
clusters are said to \emph{overlap}, denoted as
\begin{equation}
    c_i^t \circledast c_j^\tau,
\end{equation}
if they exhibit a non-negligible intersection, as determined by the underlying
cluster representation. The specific definition of overlap depends on the chosen
cluster model. Common examples include:

\begin{itemize}
    \item \textbf{Enumeration-based Representation:}
          \begin{equation}
              c_i^t \circledast c_j^\tau \iff \exists x \in \mathbb{R}^n \text{ such that } x \in c_i^t \land x \in c_j^\tau,
          \end{equation}
          indicating that the two clusters share at least one common data point.

    \item \textbf{Spherical Model Representation:}
          \begin{equation}
              c_i^t \circledast c_j^\tau \iff d(\vec{\mu}_i^t, \vec{\mu}_j^\tau) \leq r_i^t + r_j^\tau,
          \end{equation}
          where $ \vec{\mu}_i^t $ and $ \vec{\mu}_j^\tau $ denote the centroids of
          clusters $ c_i^t $ and $ c_j^\tau $, respectively, $ r_i^t $ and $ r_j^\tau $
          are their radii, and $ d(\cdot, \cdot) $ is the Euclidean distance.
\end{itemize}

The following table summarizes the formal definitions of typical cluster
transitions based on the concept of overlap:

\begin{table}[H]
    \centering
    {\small
        \begin{tabular}{|l|c|p{9cm}|}
            \hline
            \textbf{Transition} & \textbf{Notation}                                 & \textbf{Formal Definition}                                                                            \\
            \hline
            Appearance          & $\emptyset \rightarrow c_j^\tau$                  &
            $\nexists c_i^t \in C^t: c_i^t \circledast c_j^{\tau}$                                                                                                                          \\
            \hline
            Disappearance       & $c_i^t \rightarrow \emptyset$                     & $\nexists c_j^{\tau} \in C^{\tau}: c_i^t \circledast c_j^{\tau}$                                      \\
            \hline
            Split               & $c_i^t \rightarrow \{c_1^\tau, \dots, c_v^\tau\}$ & $\exists\, 1 \leq j \leq v: c_i^t \circledast c_j^\tau \land \dots \land c_i^t \circledast c_v^\tau $ \\
            \hline
            Merge               & $\{c_1^t, \dots, c_u^t\} \rightarrow c_j^\tau$    & $\exists\, 1 \leq i \leq u: c_i^t \circledast c_j^\tau \land \dots \land c_u^t \circledast c_j^\tau $ \\
            \hline
            Survival            & $c_i^t \rightarrow c_j^\tau$                      & $c_i^t \circledast c_j^\tau \land
                \nexists\, c_u^t \in C^t \setminus \{c_i^t\}: c_u^t \circledast c_j^\tau \land
            \nexists\, c_v^\tau \in C^\tau \setminus \{c_j^\tau\}: c_i^t \circledast c_v^\tau$                                                                                              \\
            \hline
        \end{tabular}
    }
    \caption{Formal definitions of cluster transitions}
    \label{table:cluster_transitions}
\end{table}

Tracking clusters in dynamic streaming environments presents several key
challenges. One of the foremost difficulties is dealing with concept drift and
evolution, where the underlying data distribution may shift gradually or
abruptly, causing clusters to move, emerge, disappear, merge, or split in
complex and unpredictable ways. The presence of noise and outliers further
complicates the problem by potentially masking genuine cluster transitions or
generating false alarms. Ambiguities often arise when clusters overlap or lie
in close proximity, making it challenging to precisely define cluster
correspondences and detect transitions accurately. Finally, the absence of
ground truth labels in streaming contexts makes it difficult to evaluate the
correctness of detected transitions and adapt algorithms to evolving cluster
structures.

\section{Dynamic Clustering}\label{sec:dynamic_clustering}

Dynamic clustering aims to provide a holistic framework for understanding and
responding to the evolution of data distributions in non-stationary
environments. By combining streaming clustering with temporal cluster tracking,
it enables both real-time analysis and retrospective insight into how clusters
emerge, change, and disappear over time.

The objective is not only to cluster incoming data efficiently, but also to
interpret and explain the structural changes in the data. This is particularly
valuable in applications where understanding data dynamics is as important as
making accurate predictions or decisions.

Dynamic clustering supports a range of analytical goals, including:

\begin{itemize}
    \item \textbf{Drift Explainability:} By analyzing the history and transformation of
          clusters, one can identify when and how concept drift occurs, and potentially
          relate it to changes in user behavior, system state, or external conditions.

    \item \textbf{Concept Monitoring and Lifecycle Analysis:} Tracking clusters over time
          provides a basis for monitoring the emergence, evolution, and retirement of
          underlying concepts or patterns in the data.

    \item \textbf{Anomaly and Event Detection:} Sudden transitions—such as abrupt merges,
          splits, or disappearances—can signal significant events or anomalies in the data stream,
          such as system faults or behavioral shifts.

    \item \textbf{Interpretability and Visualization:} Visualizing cluster evolution as
          trajectories, trees, or timelines helps domain experts make sense of complex temporal
          dynamics in an intuitive manner.
\end{itemize}

By treating clustering as a dynamic process rather than a static task, this
framework enables richer, more actionable insights in data-driven systems that
operate under change.