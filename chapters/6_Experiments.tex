\chapter{Experiments}\label{ch:experiments}
This chapter provides detailed descriptions of each dataset used in the
experiments, outlining their origins, key characteristics, and preprocessing
procedures. The evaluation spans multiple data types including tabular, image,
and text data to comprehensively assess the performance of the dynamic
clustering algorithm across different modalities. For each dataset, the results
produced by the dynamic clustering algorithm are presented and analyzed, with
particular emphasis on the interpretability and explainability of the outcomes.

The following symbols are used throughout the experimental sections to denote
key hyperparameters of the dynamic clustering algorithm: $q$ represents the
maximum number of micro-clusters, $\Delta$ denotes the window size of the
streaming clustering algorithm, and $I$ stands for the macro-clustering
triggering interval. The parameter $\varepsilon = 0.5$, that denotes the
ovelapping threshold, is chosen based on the definition of the Effective
Overlapping Score, while the confidence level $\alpha = 0.9$, that dentoes the
confidence used to compute the RMSize, is fixed to balance cluster compactness
and flexibility. Both $\varepsilon$ and $\alpha$ remain constant across all
experiments.

An extensive study on the impact of these hyperparameters on the algorithm's
performance is provided in the Appendix~\ref{sec:hyperparameter_analysis}.

The experimental sections are organized according to the nature and source of
the data: synthetic data generated under controlled conditions, data with
explicitly induced drift and real-world datasets. This structure facilitates a
systematic analysis of the algorithm's behavior under varying levels of
complexity and realism.

\section{Synthetic Data}\label{sec:sythetic_data}

To evaluate the behavior of the proposed framework under controlled
non-stationary conditions, a synthetic dataset was generated. This dataset
simulates dynamic changes in cluster structures over time and is designed to
illustrate each type of possible transition.

Each time step contains samples drawn from multivariate Gaussian distributions
in $\mathbb{R}^8$. The entire sequence consists of $20$ time steps, each
containing $100$ data points per active cluster. The generated stream evolves
smoothly, with transitions driven by interpolation and directional movement in
latent space.

The synthetic stream includes the following conceptual clusters:

\begin{itemize}
      \item \textbf{Cluster $ C_A $} (Fixed cluster): Present at every time step.
            This cluster remains stationary throughout the sequence, serving as a
            temporal anchor.

      \item \textbf{Clusters $ C_B $ and $ C_C $} (First merging pair): Initially
            positioned apart, these clusters move toward a common centroid during steps 0--4
            (merge phase), remain merged (identical means) during steps 5--9, and then split
            back to separate locations during steps 10--14.

      \item \textbf{Clusters $ C_D $ and $ C_E $} (Second merging pair): Unlike $ C_B $
            and $ C_C $, these clusters stay merged from the beginning of the stream until
            step 9. They then split into distinct clusters during steps 10--14 and gradually
            re-merge in steps 15--19.

      \item \textbf{Cluster $ C_F $} (Recurring cluster): Initially present in steps 0--4,
            this cluster disappears completely in steps 5--9. It reappears at step 10 and
            persists through step 14 before disappearing again in steps 15--19.

      \item \textbf{Cluster $ C_G $} (Appearing cluster): This cluster appears for
            the first time at step 15 and persists until the end of the stream (steps 15--19).
\end{itemize}

Each cluster is assigned a persistent label across all time steps, even during
periods of absence. The synthetic design provides explicit ground-truth
tracking information for evaluation. Additionally, random positive-definite
covariance matrices are assigned to each Gaussian component.

\begin{figure}[H]
      \centering
      \includegraphics[width=0.5\textwidth]{my_images/experiment_results/synthetic_data/data_evolution.png}
      \caption{Synthetic Data in 2D PCA evolution.}
\end{figure}

This synthetic dataset is designed as a controlled benchmark to evaluate
whether dynamic clustering algorithms can accurately capture all possible
cluster transitions under various non-stationary conditions, testing their
adaptability to evolving data.

The hyperparameters chosen for this experimental setup are as follows: $q =
      200$, $\Delta = 300$ and $I=150$.

\begin{figure}[H]
      \centering
      \includegraphics[width=1\textwidth]{my_images/experiment_results/synthetic_data/best_setting/graph.png}
      \caption{Graphical evolution of Synthtic Data Experiment.}
\end{figure}

By analyzing the cluster flow, which visualizes the temporal evolution of each
cluster, it is evident that in this controlled synthetic scenario, most of the
induced transitions are detected accurately by the proposed framework. Notably,
complex behaviors such as reappearance, re-merging, and re-splitting are
effectively captured, demonstrating the method's strong performance under these
conditions.

Nonetheless, it is important to recognize that real-world data may present
additional challenges not fully represented here. Therefore, while the results
indicate promising capabilities within this synthetic setting, further
evaluation is necessary to fully assess the method's robustness and
generalizability to more complex and noisy environments.

\section{Induced Drift data}\label{sec:induced_drift_data}

This section presents experiments on datasets derived from originally
stationary sources, where artificial drift is systematically introduced to
simulate non-stationary environments. These constructed scenarios enable
controlled evaluation of the dynamic clustering algorithm's ability to respond
to gradual changes in data distribution.

The experimental setup includes one image dataset in which brightness is
progressively altered to induce visual drift, and one text dataset where
typographical noise is incrementally applied to simulate degradation. While the
transformations differ in modality and implementation, they serve a common
purpose: to create interpretable drift patterns for assessing clustering
behavior under evolving conditions. Detailed analyses of each case are provided
in the corresponding subsections.

\subsection{Brightness drift in places data}\label{subsec:brightness}

The dataset consists of images from the Places dataset~\cite{placesdataset},
focusing on two classes: \emph{mountain} and \emph{beach}. A reference set of
500 images per class is selected. Then, for each of 5 batches, 100 images per
class are sampled. In each batch, the brightness of the images is gradually
increased using color jitter to simulate smooth and unidirectional visual
drift.

Each image is resized to $224 \times 224$ pixels before being embedded using
the CLIP model~\cite{clip} to extract high-level feature representations. CLIP
(Contrastive Language-Image Pretraining) is a neural network model trained to
connect images and natural language descriptions in a shared embedding space.
It learns to encode images into feature vectors that capture semantic content
by contrasting image-text pairs during training, enabling effective zero-shot
classification and retrieval tasks.

Subsequently, the high-dimensional CLIP embeddings are reduced to 32 dimensions
using UMAP~\cite{umap} for computational efficiency and to facilitate
clustering. UMAP (Uniform Manifold Approximation and Projection) is a nonlinear
dimension reduction technique that preserves both local and global data
structure by constructing a high-dimensional graph and optimizing a
low-dimensional embedding to maintain the data's manifold structure, allowing
effective clustering and visualization.

This experimental setup evaluates the robustness of dynamic clustering methods
under smooth brightness drift affecting the global appearance of the input
images.

The hyperparameters chosen for this experimental setup are as follows: $q =
      200$, $\Delta = 100$ and $I=50$.

\begin{figure}[H]
      \centering
      \includegraphics[width=1\textwidth]{my_images/experiment_results/image_brightness/best_setting/graph.png}
      \caption{Graphical evolution of Places-Brightness-Drift Experiment.}
\end{figure}

The results show an overall coherence between the clustering output and the
underlying evolution of the data induced by the gradual brightness drift.
Specifically, a distinct cluster emerges in the later stages of the stream,
corresponding to the `over-brightened' images that appear almost entirely
white. This can be interpreted as a meaningful splitting event from one of the
initial clusters, driven by perceptual degradation due to excessive brightness.

While the splitting is correctly identified at a high level also for cluster 0,
it is not explicitly captured by the scoring mechanism used by the algorithm,
which continues to classify the original cluster as a survivor. This limitation
may be attributed to the algorithm's sensitivity threshold or to the nature of
the drift itself, which occurs gradually and may not produce a sharply
distinguishable transition in the embedding space.

Despite this, the algorithm demonstrates a capacity to reflect the structural
evolution of the data, with new clusters forming in response to accumulated
visual changes. The experiment illustrates that, although dynamic clustering
may not always detect smooth transformations as discrete events, it can still
adapt to continuous appearance shifts.

Additionally, the visual nature of the stream and the resulting cluster
transitions provide a form of human-interpretable drift analysis: the
appearance of distinct clusters corresponding to perceptually different image
groups offers an intuitive and explainable representation of how the data is
evolving. This highlights the potential of dynamic clustering not only as an
unsupervised tracking tool, but also as a mechanism for supporting
interpretability in scenarios involving smooth or continuous distributional
shifts.

\begin{figure}[H]
      \centering
      \includegraphics[width=0.75\textwidth]{my_images/experiment_results/image_brightness/best_setting/selected_rsamples.png}
      \caption{Representative Samples of clusters of Brightness Data.}
\end{figure}

\begin{figure}[H]
      \centering
      \includegraphics[width=1\textwidth]{my_images/experiment_results/image_brightness/best_setting/selected_spatial.png}
      \caption{Spatial Evolution of clusters of Brightness Data.}
\end{figure}

\subsection{Typographical drift}\label{subsec:typographical_drift}
To evaluate the robustness of dynamic clustering under linguistic noise, a
subset of the 20 Newsgroups dataset~\cite{20newsgroups} was used, consisting of
four categories: \emph{rec.autos}, \emph{comp.graphics}, \emph{sci.med}, and
\emph{rec.sport.baseball}. These categories represent distinct topics, enabling
clear cluster structure in the original space.

The experimental stream was constructed in two phases. Initially, 100 clean
documents per class were sampled to serve as a reference distribution.
Subsequently, for each of three drift stages, 278 documents per class were
sampled and subjected to varying levels of typographical corruption. The
corruption was introduced using a synthetic noise function designed to simulate
realistic typing errors through random character-level modifications, including
insertions, deletions, replacements, and swaps. The severity of corruption was
progressively increased during the experiment, resulting in more aggressive
drift at later stages.

Each text sample was embedded into a 384-dimensional semantic vector using the
\texttt{all\allowbreak-MiniLM\allowbreak-L6\allowbreak-v2}
model~\cite{sentence-transformers}, a transformer-based sentence encoder
optimized for efficient semantic representation. Dimensionality was
subsequently reduced to 32 using UMAP~\cite{umap} to facilitate clustering and
reduce computational overhead.

This setup emulates a scenario in which textual input quality degrades
progressively while preserving underlying topical structure, offering a
controlled test for evaluating the dynamic clustering algorithm to text data.

The hyperparameters chosen for this experimental setup are as follows: $q =
      200$, $\Delta = 100$ and $I=100$.

\begin{figure}[H]
      \centering
      \includegraphics[width=1\textwidth]{my_images/experiment_results/text_typos/best_setting/graph.png}
      \caption{Graphical evolution of Text Typos Experiment.}
\end{figure}

The results indicate that the drift-induced cluster initially emerges as a
split from \emph{cluster 2}. Throughout the period in which the textual content
remains relatively intact, the algorithm successfully identifies a sequence of
merging, resplitting, and remerging transitions. This demonstrates the method's
ability to track gradual and reversible changes within the data distribution.

Only when the text becomes significantly corrupted does the algorithm recognize
a fully distinct new cluster, reflecting a more pronounced shift in the
underlying data characteristics. This behavior suggests that the dynamic
clustering approach is sensitive to the degree of drift and can differentiate
between subtle perturbations and more substantial distributional changes.

Overall, these findings highlight the algorithm's capacity to model complex
temporal cluster dynamics in scenarios where data evolves progressively,
supporting its applicability to real-world tasks involving incremental or
partial data degradation, including applications in text data where content may
gradually become corrupted or altered.

\begin{figure}[H]
      \centering
      \includegraphics[width=1\textwidth]{my_images/experiment_results/text_typos/best_setting/selected_spatial.png}
      \caption{Graphical evolution of Text Typos Clusters.}
\end{figure}

\begin{table}[H]
    \centering
    \scriptsize
    \begin{tabularx}{\textwidth}{c|l|X}
        \hline
        \textbf{Cluster} & \textbf{Intepretation} & \textbf{Text Sample}                                                                                                                                                                                                                                                                             \\
        \hline
        0                & sci.med                & \emph{Ihf antne has any informatiin on thi deficiency would veyr ggreatly apprecaite a response hereq or preferably by Emaul. All I know at tihs opin is a defjciencg can caue smyoglovin to be releasfd,a dn in times of stresz ad hgh ambivent temperatursy could caussez grenal failure. x} \\
        \hline
        1                & rec.sport.baseball     & \emph{Bobby Bonilla supposedly use the word 'faggot' when he got mad at that author in the clubhouse. Should he be banned from baseball for a year like Schott?}                                                                                                                               \\
        \hline
        3                & comp.graphics          & \emph{Ihf antne has a ljst if cimpanies oing dagta isualizatoing (software or ahdwzre)I woild liek toheapzrf rom tyem. Thank.s - - rs --}                                                                                                                                                      \\
        \hline
        2                & rec.autos              & \emph{Cup holders (driving is an importantant enough undertaking) Cellular phones and mobile fax machines (see above) Vanity mirrors on the driver's side. Ashtrays (smokers seem to think it's just fine to use the road) Fake convertible roofs and vinyl roofs. Any gold trim.}             \\
        \hline
        6                & Typos                  & \emph{Thhe qystion is nkt wgethfe your adio wigll e stolen. Tghe question isw en your ryaeio will bestolen.}                                                                                                                                                                                   \\
        \hline
    \end{tabularx}
    \caption{Representative Samples at iteration $t=3300$ of Text Typos clusters.}
\end{table}

\section{Real Data}\label{sec:real_data}

This section presents experiments conducted on real-world datasets to evaluate
the practical applicability and robustness of the dynamic clustering algorithm
in realistic scenarios. Unlike synthetic or induced drift datasets, where the
nature and evolution of the drift are controlled and known, real data
introduces natural variability, noise, and less predictable temporal patterns,
offering a more challenging and representative testbed.

Two distinct datasets are considered: one tabular dataset from the domain of
network intrusion detection, and one image dataset capturing gradual visual
variations in a collection of fruit photographs. These datasets are selected to
cover different data modalities and to assess the algorithm's effectiveness in
modeling cluster evolution where the underlying ground truth may be incomplete
or ambiguous.

By applying the dynamic clustering method to these real datasets, the aim is to
observe how well the algorithm identifies meaningful structural changes in the
data over time and whether it provides interpretable insights into temporal
patterns without artificial drift induction.

\subsection{KDD Cup 99}\label{subsec:kdd_data}
The KDD Cup 1999 dataset~\cite{kdd99dataset} is a widely recognized benchmark
for evaluating intrusion detection systems (IDS) and machine learning
techniques in the cybersecurity domain. It was developed from network traffic
data collected during the DARPA 1998 Intrusion Detection Evaluation Program.
The dataset comprises millions of connection records, each labeled as either a
normal connection or one of 22 different attack types. Each instance is
represented by 41 features, which include both continuous numerical values and
symbolic (categorical) attributes

A widely used and more computationally manageable alternative to the full
dataset is the \textbf{10\% subset}, which contains approximately 494,021
records. This subset maintains the same feature structure and class
distribution as the original dataset. In this study, the 10\% subset is
employed to facilitate the application of dynamic clustering algorithms and to
ensure tractable processing times in an evolving data stream scenario.

To prepare the dataset for clustering, all categorical features were removed,
resulting in a set of 32 continuous numerical features. Subsequently, a Min-Max
normalization was applied to scale each feature into the range $[0,1]$,
standardizing the feature space and mitigating the impact of differing value
ranges on clustering algorithm.

The KDD99 dataset has been extensively used in the literature to evaluate the
performance of streaming and dynamic clustering algorithms in evolving
environments. Notable examples include \textsc{CluStream}~\cite{clustream},
\textsc{DenStream}~\cite{denstream}, \textsc{D-Stream}~\cite{d_stream}, and
\textsc{Stream\-KM++}~\cite{stream_km_plus_plus}.

In the context of dynamic clustering, each cluster can be interpreted as a
group of network connections exhibiting similar behavioral patterns. These
clusters may correspond to distinct categories of network activity such as:
Normal traffic, Denial of Service DoS (e.g., \emph{smurf}, \emph{neptune}),
User to Root U2R (e.g., \emph{buffer\_overflow}), Remote to Local R2L (e.g.,
\emph{guess\_passwd}) or Probing (e.g., \emph{nmap}, \emph{satan}).

The hyperparameters chosen for this experimental setup are as follows: $q =
      100$, $\Delta = 5000$ and $I=5000$.

\begin{figure}[H]
      \centering
      \includegraphics[width=1\textwidth]{my_images/experiment_results/kdd_data/best_setting/graph.png}
      \caption{Graphical evolution of KDD-CUP-99 clusters.}
\end{figure}

The results reveal a generally unstable behavior during the first iterations,
with a high number of clusters being created over time. These clusters often
lack continuity, showing minimal overlap with preceding or succeeding clusters,
which indicates a limited temporal coherence in the output.

However, this apparent instability partially reflects the inherent variability
in the dataset itself. As shown in the labels history
(Figure~\ref{fig:kdd_labels_history}), the data stream exhibits alternating
phases of stability and rapid change. In particular, during periods when the
stream consists predominantly of a single class label, the clustering results
tend to be more stable, often producing persistent clusters. Conversely, in
segments where the stream alternates frequently between multiple labels, the
clustering output becomes more fragmented and unstable.

\begin{figure}[H]
      \centering
      \includegraphics[width=1\textwidth]{my_images/experiment_results/kdd_data/label_history.png}
      \caption{Labels history of KDD-CUP-99 Dataset.}
      \label{fig:kdd_labels_history}
\end{figure}

This behavior suggests that the dynamic clustering algorithm is responsive to
shifts in the underlying data distribution, even if it does not maintain strong
temporal consistency throughout. The instability in the results may also be
attributed to limitations of the macro-clustering component, which could be
improved to enhance robustness. Nonetheless, the algorithm is capable of
distinguishing phases of relative stability and transition in the data stream,
which is a desirable property for applications in dynamic and evolving
environments.

%TODO: add plot of the macroclustering statisticcs to show that maybe macroclustering is not too good
\subsection{Fruits}\label{subsec:fruits}

A collected dataset of 2,420 RGB images of tangerines, each resized to $100
      \times 100$ pixels, was used to investigate dynamic clustering for images under
real-world conditions.

To isolate the foreground objects and reduce background noise, automatic
background segmentation was applied using the GrabCut algorithm~\cite{grabcut}.
This technique leverages an iterative graph-cut procedure initialized with a
fixed bounding box to distinguish foreground from background, resulting in
4-channel RGBA images with transparent backgrounds.

Following segmentation, each image was encoded using a pretrained ResNet-50
convolutional neural network~\cite{resnet}, producing high-dimensional visual
feature embeddings. These embeddings were subsequently reduced to 64 dimensions
using UMAP~\cite{umap} to facilitate efficient clustering and visualization.

The hyperparameters chosen for this experimental setup are as follows: $q =
      200$, $\Delta = 50$, and $I = 50$.

To simulate a realistic deployment scenario, the first 200 images of the
dataset are treated as reference data, representing the initial state of the
system. The remaining 2,220 images are considered production data,
progressively ingested by the dynamic clustering pipeline.

\begin{figure}[H]
      \centering
      \includegraphics[width=1\textwidth]{my_images/experiment_results/fruits/best_setting/graph.png}
      \caption{Graphical evolution of Fruits Images Experiment.}
\end{figure}

The results indicate that the algorithm effectively captures the appearance of
new tangerine types over time. Visual inspection of the clustered images
reveals distinct characteristics associated with each cluster:

\begin{itemize}
      \item \emph{Clusters 0 and 2}: Round yellow-green tangerines
      \item \emph{Cluster 1}: Oval-shaped yellow-green tangerines
      \item \emph{Cluster 3}: Round orange tangerines
      \item \emph{Cluster 4}: Oval-shaped orange tangerines
\end{itemize}

Similar to the previous experiment, the algorithm identifies only one of the
two expected splits, which may be attributed to the sensitivity threshold of
the method. This partial detection does not detract significantly from the
results, as the emergence of the orange tangerines is still successfully
recognized as a meaningful transition in the data stream.

As the data stream progresses, the dataset exhibits a gradual evolution in
visual characteristics. Initially, the images consist primarily of green and
yellow tangerines, while later segments include increasingly orange and reddish
specimens. In parallel, subtle variations in shape further contribute to the
evolving distribution. These gradual changes introduce a form of distributional
drift in the visual domain, which the dynamic clustering method successfully
tracks over time.

The effectiveness of the approach in capturing such shifts underlines its
applicability to real-world scenarios where visual data distributions evolve
slowly and nonlinearly.

\begin{figure}[H]
      \centering
      \includegraphics[width=1\textwidth]{my_images/experiment_results/fruits/best_setting/selected_spatial.png}
      \caption{Spatial Evolution of clusters of Fruit Images.}
\end{figure}

\begin{figure}[H]
      \centering
      \includegraphics[width=1\textwidth]{my_images/experiment_results/fruits/best_setting/selected_rsamples.png}
      \caption{Representative Samples of clusters of Fruit Images.}
\end{figure}