\chapter{Conclusions and Future Developments}\label{ch:conclusions}

\section{Overview}\label{sec:overview}

This thesis explored the problem of dynamic clustering in non-stationary data
streams, where the underlying distribution evolves over time. In such settings,
traditional static clustering methods fall short due to their inability to
adapt to temporal changes and recurring structures.

To address these challenges, a modular and adaptive clustering framework was
developed, designed to maintain temporally coherent and interpretable cluster
representations as the data evolves. Central to this framework is a typology of
cluster transitions, such as \emph{survival}, \emph{absorption}, and
\emph{splitting}, which enables a fine-grained analysis of structural changes
between time steps. This was further extended to capture long-term recurrences
through transitions such as \emph{reappearance}, \emph{remerge}, and
\emph{resplit}, identified via graph-based analysis and temporal matching
heuristics.

Additionally, novel spatial proximity scores were introduced to measure the
degree of overlap between clusters. In particular, the \emph{Spherical
    Overlapping Score (SOS)} and the \emph{Effective Overlapping Score} provide
geometry-aware, scale-invariant criteria to support transition detection. These
metrics enhance the robustness and interpretability of cluster tracking,
especially in scenarios involving gradual drift or recurring patterns.

The overall architecture is designed for incremental use and supports
memory-efficient summarization, visualization, and cluster-level reporting.
Experiments demonstrated the effectiveness of the proposed framework in
capturing the dynamic evolution of clustering structures over time, as well as
its interpretability in practical scenarios.

In summary, the thesis contributes a principled and interpretable approach to
dynamic clustering, enabling the analysis of evolving structures in streaming
data through a rich set of transition types and overlap metrics.

\section{Future Developments}\label{sec:future_developments}

While the proposed framework provides a structured and interpretable approach
to dynamic clustering, several limitations remain, suggesting opportunities for
future work.

A key area for extension involves the representation of density-based
clustering results. Algorithms such as DBSCAN and HDBSCAN, which are commonly
used in practice, do not yield explicit centroids, making the existing overlap
metrics unsuitable. Future developments could focus on defining compact
representations specific to density-based clusters and designing new overlap
scores that reflect their spatial and structural characteristics.

The modular architecture of the proposed framework presents another avenue for
future enhancement. Each core component can be independently modified or
extended. In particular, future work could explore more advanced triggering
strategies, potentially informed by drift detection methods, to better identify
when significant changes in the data stream warrant re-clustering. This design
flexibility also enables the integration of alternative clustering techniques
and distance functions, allowing the framework to be adapted to a wide range of
application domains.

Another promising direction is the forecasting of cluster transitions.
Transition prediction would move the system beyond reactive analysis toward
proactive modeling. Since the current framework already produces rich temporal
metadata describing cluster dynamics, it offers a strong foundation for
learning temporal patterns and anticipating future changes in the clustering
structure.

